% Kommentare für den Editor (TexWorks/TexMakerX)
% !TeX encoding   = utf8
% !TeX spellcheck = de-DE

% Dokumentenklasse (Koma Script) -----------------------------------------
\documentclass[%
   %draft,     % Entwurfsstadium
   final,      % fertiges Dokument
   paper=a4, paper=portrait, pagesize=auto, % Papier Einstellungen
   fontsize=11pt, % Schriftgröße
   ngerman, % Sprache
 ]{scrartcl} % Classes: scrartcl, scrreprt, scrbook

% ~~~~~~~~~~~~~~~~~~~~~~~~~~~~~~~~~~~~~~~~~~~~~~~~~~~~~~~~~~~~~~~~~~~~~~~~
% encoding
% ~~~~~~~~~~~~~~~~~~~~~~~~~~~~~~~~~~~~~~~~~~~~~~~~~~~~~~~~~~~~~~~~~~~~~~~~

% Encoding der Dateien (sonst funktionieren Umlaute nicht)
\usepackage[utf8]{inputenc}

% Encoding der Verzeichnisse (für Pfade mit Umlauten und Leerzeichne)
\usepackage[%
   extendedchars, encoding, multidot, space,
   filenameencoding=latin1, % Windows XP, Vista, 7
   % filenameencoding=utf8,   % Linux, OS X
]{grffile}

% ~~~~~~~~~~~~~~~~~~~~~~~~~~~~~~~~~~~~~~~~~~~~~~~~~~~~~~~~~~~~~~~~~~~~~~~~
% Pakete und Stile
% ~~~~~~~~~~~~~~~~~~~~~~~~~~~~~~~~~~~~~~~~~~~~~~~~~~~~~~~~~~~~~~~~~~~~~~~~
% Schriften
% ~~~~~~~~~~~~~~~~~~~~~~~~~~~~~~~~~~~~~~~~~~~~~~~~~~~~~~~~~~~~~~~~~~~~~~~~
% Fonts Fonts Fonts
% ~~~~~~~~~~~~~~~~~~~~~~~~~~~~~~~~~~~~~~~~~~~~~~~~~~~~~~~~~~~~~~~~~~~~~~~~

% immer laden:
\usepackage[T1]{fontenc} % T1 Schrift Encoding
\usepackage{textcomp}	 % Zusätzliche Symbole (Text Companion font extension)

% ~~~~~~~~~~~~~~~~~~~~~~~~~~~~~~~~~~~~~~~~~~~~~~~~~~~~~~~~~~~~~~~~~~~~~~~~
% Symbole
% ~~~~~~~~~~~~~~~~~~~~~~~~~~~~~~~~~~~~~~~~~~~~~~~~~~~~~~~~~~~~~~~~~~~~~~~~

\usepackage{amssymb}
\usepackage{mathcomp}


%% ==== Zusammengesetzte Schriften  (Sans + Serif) =======================

%% - Latin Modern
\usepackage{lmodern}
%% -------------------

%% - Bera Schriften
%\usepackage{bera}
%% -------------------

%% - Times, Helvetica, Courier (Word Standard...)
%\usepackage{mathptmx}
%\usepackage[scaled=.90]{helvet}
%\usepackage{courier}
%% -------------------

%% - Palantino , Helvetica, Courier
%\usepackage{mathpazo}
%\usepackage[scaled=.95]{helvet}
%\usepackage{courier}
%% -------------------

%% - Charter, Bera Sans
%\usepackage{charter}\linespread{1.05}
%\renewcommand{\sfdefault}{fvs}
%\usepackage[charter]{mathdesign}



%%%% =========== Typewriter =============

%\usepackage{courier}                   %% --- Courier
%\renewcommand{\ttdefault}{cmtl}        %% --- CmBright Typewriter Font
%\usepackage[%                          %% --- Luxi Mono (Typewriter)
%   scaled=0.9
%]{luximono}



% Pakete Laden
% ~~~~~~~~~~~~~~~~~~~~~~~~~~~~~~~~~~~~~~~~~~~~~~~~~~~~~~~~~~~~~~~~~~~~~~~~
% These packages must be loaded before all others
% (primarily because they are required by other packages)
% ~~~~~~~~~~~~~~~~~~~~~~~~~~~~~~~~~~~~~~~~~~~~~~~~~~~~~~~~~~~~~~~~~~~~~~~~
\usepackage{calc}
\usepackage{fixltx2e}	% Fix known LaTeX2e bugs

\usepackage[ngerman]{babel} 			% Sprache
\usepackage[dvipsnames, table]{xcolor} 	% Farben

% ~~~~~~~~~~~~~~~~~~~~~~~~~~~~~~~~~~~~~~~~~~~~~~~~~~~~~~~~~~~~~~~~~~~~~~~~
% Bilder, Gleitumgebungen und Platzierung
% ~~~~~~~~~~~~~~~~~~~~~~~~~~~~~~~~~~~~~~~~~~~~~~~~~~~~~~~~~~~~~~~~~~~~~~~~
\graphicspath{{image/}}
\usepackage[]{graphicx}					% Graphiken
\usepackage{epstopdf}		% konvertiert eps in pdf

% provides new floats and enables H float modifier option
\usepackage{float}
% Floats immer erst nach der Referenz setzen
\usepackage{flafter}
% Alel Floats werden vor der nächsten section ausgegeben
\usepackage[section]{placeins} 
%

% ~~~~~~~~~~~~~~~~~~~~~~~~~~~~~~~~~~~~~~~~~~~~~~~~~~~~~~~~~~~~~~~~~~~~~~~~
% Beschriftungen (captions)
% ~~~~~~~~~~~~~~~~~~~~~~~~~~~~~~~~~~~~~~~~~~~~~~~~~~~~~~~~~~~~~~~~~~~~~~~~

\usepackage{caption}
\usepackage{subcaption}

% ~~~~~~~~~~~~~~~~~~~~~~~~~~~~~~~~~~~~~~~~~~~~~~~~~~~~~~~~~~~~~~~~~~~~~~~~
% Math
% ~~~~~~~~~~~~~~~~~~~~~~~~~~~~~~~~~~~~~~~~~~~~~~~~~~~~~~~~~~~~~~~~~~~~~~~~

% Base Math Package
\usepackage[fleqn]{amsmath} 
% Warnt bei Benutzung von Befehlen die mit amsmath inkompatibel sind.
\usepackage[all, error]{onlyamsmath}

% ~~~~~~~~~~~~~~~~~~~~~~~~~~~~~~~~~~~~~~~~~~~~~~~~~~~~~~~~~~~~~~~~~~~~~~~~
% Science
% ~~~~~~~~~~~~~~~~~~~~~~~~~~~~~~~~~~~~~~~~~~~~~~~~~~~~~~~~~~~~~~~~~~~~~~~~

% Einheiten und Zahlenformatierung
\usepackage{siunitx}

% ~~~~~~~~~~~~~~~~~~~~~~~~~~~~~~~~~~~~~~~~~~~~~~~~~~~~~~~~~~~~~~~~~~~~~~~~
% Tables (Tabular)
% ~~~~~~~~~~~~~~~~~~~~~~~~~~~~~~~~~~~~~~~~~~~~~~~~~~~~~~~~~~~~~~~~~~~~~~~~

\usepackage{booktabs}
\usepackage{ltxtable} % Longtable + tabularx

% ~~~~~~~~~~~~~~~~~~~~~~~~~~~~~~~~~~~~~~~~~~~~~~~~~~~~~~~~~~~~~~~~~~~~~~~~
% text related packages
% ~~~~~~~~~~~~~~~~~~~~~~~~~~~~~~~~~~~~~~~~~~~~~~~~~~~~~~~~~~~~~~~~~~~~~~~~

\usepackage{url}            % Befehl \url{...}
\usepackage{enumitem}		% Kompakte Listen

% Neue Befehle: \Centering, \RaggedLeft, and \RaggedRight, ... 
\usepackage{ragged2e}


% ~~~~~~~~~~~~~~~~~~~~~~~~~~~~~~~~~~~~~~~~~~~~~~~~~~~~~~~~~~~~~~~~~~~~~~~~
% Citations
% ~~~~~~~~~~~~~~~~~~~~~~~~~~~~~~~~~~~~~~~~~~~~~~~~~~~~~~~~~~~~~~~~~~~~~~~~

%\usepackage[
%	style=alphabetic, % Loads the bibliography and the citation style 
%	natbib=true, % define natbib compatible cite commands
%]{biblatex}	
% Other options:
%	style=numeric, % 
%	style=numeric-comp,    % [1–3, 7, 8]
%	style=numeric-verb,    % [2]; [5]; [6]


% ~~~~~~~~~~~~~~~~~~~~~~~~~~~~~~~~~~~~~~~~~~~~~~~~~~~~~~~~~~~~~~~~~~~~~~~~
% layout packages
% ~~~~~~~~~~~~~~~~~~~~~~~~~~~~~~~~~~~~~~~~~~~~~~~~~~~~~~~~~~~~~~~~~~~~~~~~
%
% Befehle für 1,5 und 2 zeilig: 
% \singlespacing, \onehalfspacing und \doublespacing
\usepackage{setspace}

% ~~~~~~~~~~~~~~~~~~~~~~~~~~~~~~~~~~~~~~~~~~~~~~~~~~~~~~~~~~~~~~~~~~~~~~~~
% Kopf und Fusszeile
% ~~~~~~~~~~~~~~~~~~~~~~~~~~~~~~~~~~~~~~~~~~~~~~~~~~~~~~~~~~~~~~~~~~~~~~~~

% Kopf und Fusszeile mit scrpage2 einstellen
\usepackage[automark, komastyle, nouppercase]{scrpage2}

% ~~~~~~~~~~~~~~~~~~~~~~~~~~~~~~~~~~~~~~~~~~~~~~~~~~~~~~~~~~~~~~~~~~~~~~~~
% pdf packages
% ~~~~~~~~~~~~~~~~~~~~~~~~~~~~~~~~~~~~~~~~~~~~~~~~~~~~~~~~~~~~~~~~~~~~~~~~

% Include pages from external PDF documents in LaTeX documents
\usepackage{pdfpages} 

% Optischer Randausgleich mit pdfTeX
\usepackage{microtype}

\usepackage[unicode]{hyperref}

\usepackage{listings}

% Einstellungen und Layoutstile
% ~~~~~~~~~~~~~~~~~~~~~~~~~~~~~~~~~~~~~~~~~~~~~~~~~~~~~~~~~~~~~~~~~~~~~~~~
% Colors
% ~~~~~~~~~~~~~~~~~~~~~~~~~~~~~~~~~~~~~~~~~~~~~~~~~~~~~~~~~~~~~~~~~~~~~~~~
\definecolor{sectioncolor}{RGB}{0, 0, 0}     % black

% ~~~~~~~~~~~~~~~~~~~~~~~~~~~~~~~~~~~~~~~~~~~~~~~~~~~~~~~~~~~~~~~~~~~~~~~~
% text related 
% ~~~~~~~~~~~~~~~~~~~~~~~~~~~~~~~~~~~~~~~~~~~~~~~~~~~~~~~~~~~~~~~~~~~~~~~~

%% style of URL
\urlstyle{tt}


% Keine hochgestellten Ziffern in der Fussnote (KOMA-Script-spezifisch):
\deffootnote{1.5em}{1em}{\makebox[1.5em][l]{\thefootnotemark}}

% Limit space of footnotes to 10 lines
\setlength{\dimen\footins}{10\baselineskip}

% prevent continuation of footnotes 
% at facing page
\interfootnotelinepenalty=10000 

% ~~~~~~~~~~~~~~~~~~~~~~~~~~~~~~~~~~~~~~~~~~~~~~~~~~~~~~~~~~~~~~~~~~~~~~~~
% Science
% ~~~~~~~~~~~~~~~~~~~~~~~~~~~~~~~~~~~~~~~~~~~~~~~~~~~~~~~~~~~~~~~~~~~~~~~~

\sisetup{%
	mode = math, detect-family, detect-weight,	
	exponent-product = \cdot,
	number-unit-separator=\text{\,},
	output-decimal-marker={,},
}

% ~~~~~~~~~~~~~~~~~~~~~~~~~~~~~~~~~~~~~~~~~~~~~~~~~~~~~~~~~~~~~~~~~~~~~~~~
% Citations / Style of Bibliography
% ~~~~~~~~~~~~~~~~~~~~~~~~~~~~~~~~~~~~~~~~~~~~~~~~~~~~~~~~~~~~~~~~~~~~~~~~

% Kommentar entfernene wenn biblatex geladen wird
% \IfPackageLoaded{biblatex}{%
	\ExecuteBibliographyOptions{%
%--- Backend --- --- ---
	backend=bibtex,  % (bibtex, bibtex8, biber)
	bibwarn=true, %
	bibencoding=ascii, % (ascii, inputenc, <encoding>)
%--- Sorting --- --- ---
	sorting=nty, % Sort by name, title, year.
	% other options: 
	% nty        Sort by name, title, year.
	% nyt        Sort by name, year, title.
	% nyvt       Sort by name, year, volume, title.
	% anyt       Sort by alphabetic label, name, year, title.
	% anyvt      Sort by alphabetic label, name, year, volume, title.
	% ynt        Sort by year, name, title.
	% ydnt       Sort by year (descending), name, title.
	% none       Do not sort at all. All entries are processed in citation order.
	% debug      Sort by entry key. This is intended for debugging only.
	%
	sortcase=true,
	sortlos=los, % (bib, los) The sorting order of the list of shorthands
	sortcites=false, % do/do not sort citations according to bib	
%--- Dates --- --- ---
	date=comp,  % (short, long, terse, comp, iso8601)
%	origdate=
%	eventdate=
%	urldate=
%	alldates=
	datezeros=true, %
	dateabbrev=true, %
%--- General Options --- --- ---
	maxnames=1,
	minnames=1,
%	maxbibnames=99,
%	maxcitenames=1,
%	autocite= % (plain, inline, footnote, superscript) 
	autopunct=true,
	language=auto,
	babel=none, % (none, hyphen, other, other*)
	block=none, % (none, space, par, nbpar, ragged)
	notetype=foot+end, % (foot+end, footonly, endonly)
	hyperref=true, % (true, false, auto)
	backref=true,
	backrefstyle=three, % (none, three, two, two+, three+, all+)
	backrefsetstyle=setonly, %
	indexing=false, % 
	% options:
	% true       Enable indexing globally.
	% false      Disable indexing globally.
	% cite       Enable indexing in citations only.
	% bib        Enable indexing in the bibliography only.
	refsection=none, % (part, chapter, section, subsection)
	refsegment=none, % (none, part, chapter, section, subsection)
	abbreviate=true, % (true, false)
	defernumbers=false, % 
	punctfont=false, % 
	arxiv=abs, % (ps, pdf, format)	
%--- Style Options --- --- ---	
% The following options are provided by the standard styles
	isbn=false,%
	url=false,%
	doi=false,%
	eprint=false,%	
	}%	
	
	% change alpha label to be without +	
	\renewcommand*{\labelalphaothers}{}
	
	% change 'In: <magazine>" to "<magazine>"
	\renewcommand*{\intitlepunct}{}
	\DefineBibliographyStrings{german}{in={}}
	
	% make names capitalized \textsc{}
	\renewcommand{\mkbibnamefirst}{\textsc}
	\renewcommand{\mkbibnamelast}{\textsc}
	
	% make volume and number look like 
	% 'Bd. 33(14): '
	\renewbibmacro*{volume+number+eid}{%
	  \setunit{\addcomma\space}%
	  \bibstring{volume}% 
	  \setunit{\addspace}%
	  \printfield{volume}%
	  \iffieldundef{number}{}{% 
	    \printtext[parens]{%
	      \printfield{number}%
	    }%
	  }%
	  \setunit{\addcomma\space}%
	  \printfield{eid}
	  %\setunit{\addcolon\space}%
	  }	

	% <authors>: <title>
	\renewcommand*{\labelnamepunct}{\addcolon\space}
	% make ': ' before pages
	\renewcommand*{\bibpagespunct}{\addcolon\space}
	% names delimiter ';' instead of ','
	%\renewcommand*{\multinamedelim}{\addsemicolon\space}

	% move date before issue
	\renewbibmacro*{journal+issuetitle}{%
	  \usebibmacro{journal}%
	  \setunit*{\addspace}%
	  \iffieldundef{series}
	    {}
	    {\newunit
	     \printfield{series}%
	     \setunit{\addspace}}%
	  %
	  \usebibmacro{issue+date}%
	  \setunit{\addcolon\space}%
	  \usebibmacro{issue}%
	  \setunit{\addspace}%
	  \usebibmacro{volume+number+eid}%
	  \newunit}

	% print all names, even if maxnames = 1
	\DeclareCiteCommand{\citeauthors}
	  {
	   \defcounter{maxnames}{1000}
	   \boolfalse{citetracker}%
	   \boolfalse{pagetracker}%
	   \usebibmacro{prenote}}
	  {\ifciteindex
	     {\indexnames{labelname}}
	     {}%
	   \printnames{labelname}}
	  {\multicitedelim}
	  {\usebibmacro{postnote}}

}%

% ~~~~~~~~~~~~~~~~~~~~~~~~~~~~~~~~~~~~~~~~~~~~~~~~~~~~~~~~~~~~~~~~~~~~~~~~
% figures, placement, floats and captions
% ~~~~~~~~~~~~~~~~~~~~~~~~~~~~~~~~~~~~~~~~~~~~~~~~~~~~~~~~~~~~~~~~~~~~~~~~

% Make float placement easier
\renewcommand{\floatpagefraction}{.75} % vorher: .5
\renewcommand{\textfraction}{.1}       % vorher: .2
\renewcommand{\topfraction}{.8}        % vorher: .7
\renewcommand{\bottomfraction}{.5}     % vorher: .3
\setcounter{topnumber}{3}        % vorher: 2
\setcounter{bottomnumber}{2}     % vorher: 1
\setcounter{totalnumber}{5}      % vorher: 3

%% ~~~ Captions ~~~~~~~~~~~~~~~~~~~~~~~~~~~~~~~~~~~~~~~~~~~~~~~~~~~~~~~~~~
% Style of captions
\DeclareCaptionStyle{captionStyleTemplateDefault}
[ % single line captions
   justification = centering
]
{ % multiline captions
% -- Formatting
   format      = plain,  % plain, hang
   indention   = 0em,    % indention of text 
   labelformat = default,% default, empty, simple, brace, parens
   labelsep    = colon,  % none, colon, period, space, quad, newline, endash
   textformat  = simple, % simple, period
% -- Justification
   justification = justified, %RaggedRight, justified, centering
   singlelinecheck = true, % false (true=ignore justification setting in single line)
% -- Fonts
   labelfont   = {small,bf},
   textfont    = {small,rm},
% valid values:
% scriptsize, footnotesize, small, normalsize, large, Large
% normalfont, ip, it, sl, sc, md, bf, rm, sf, tt
% singlespacing, onehalfspacing, doublespacing
% normalcolor, color=<...>
%
% -- Margins and further paragraph options
   margin = 10pt, %.1\textwidth,
   % width=.8\linewidth,
% -- Skips
   skip     = 10pt, % vertical space between the caption and the figure
   position = auto, % top, auto, bottom
% -- Lists
   % list=no, % suppress any entry to list of figure 
   listformat = subsimple, % empty, simple, parens, subsimple, subparens
% -- Names & Numbering
   % figurename = Abb. %
   % tablename  = Tab. %
   % listfigurename=
   % listtablename=
   % figurewithin=chapter
   % tablewithin=chapter
%-- hyperref related options
	hypcap=true, % (true, false) 
	% true=all hyperlink anchors are placed at the 
	% beginning of the (floating) environment
	%
	hypcapspace=0.5\baselineskip
}

% apply caption style
\captionsetup{
	style = captionStyleTemplateDefault % base
}

% Predefinded skip setup for different floats
\captionsetup[table]{position=top}
\captionsetup[figure]{position=bottom}


% options for subcaptions
\captionsetup[sub]{ %
	style = captionStyleTemplateDefault, % base
	skip=6pt,
	margin=5pt,
	labelformat = parens,% default, empty, simple, brace
	labelsep    = space,
	list=false,
	hypcap=false
}

% ~~~~~~~~~~~~~~~~~~~~~~~~~~~~~~~~~~~~~~~~~~~~~~~~~~~~~~~~~~~~~~~~~~~~~~~~
% layout 
% ~~~~~~~~~~~~~~~~~~~~~~~~~~~~~~~~~~~~~~~~~~~~~~~~~~~~~~~~~~~~~~~~~~~~~~~~


%% Paragraph Separation =================================
\KOMAoptions{%
   parskip=absolute, % do not change indentation according to fontsize
   parskip=false     % indentation of 1em
   % parskip=half    % parksip of 1/2 line 
}%

%% line spacing =========================================
%\onehalfspacing	% 1,5-facher Abstand
%\doublespacing		% 2-facher Abstand

%% page layout ==========================================

\raggedbottom     % Variable Seitenhoehen zulassen

% Koma Script text area layout
\KOMAoptions{%
   DIV=11,% (Size of Text Body, higher values = greater textbody)
   BCOR=5mm% (Bindekorrektur)
}%

%%% === Page Layout  Options ===
\KOMAoptions{% (most options are for package typearea)
   % twoside=true, % two side layout (alternating margins, standard in books)
   twoside=false, % single side layout 
   %
   headlines=2.1,%
}%

%\KOMAoptions{%
%      headings=noappendixprefix % chapter in appendix as in body text
%      ,headings=nochapterprefix  % no prefix at chapters
%      % ,headings=appendixprefix   % inverse of 'noappendixprefix'
%      % ,headings=chapterprefix    % inverse of 'nochapterprefix'
%      % ,headings=openany   % Chapters start at any side
%      % ,headings=openleft  % Chapters start at left side
%      ,headings=openright % Chapters start at right side      
%}%


% reloading of typearea, necessary if setting of spacing changed
\typearea[current]{last}

% ~~~~~~~~~~~~~~~~~~~~~~~~~~~~~~~~~~~~~~~~~~~~~~~~~~~~~~~~~~~~~~~~~~~~~~~~
% Titlepage
% ~~~~~~~~~~~~~~~~~~~~~~~~~~~~~~~~~~~~~~~~~~~~~~~~~~~~~~~~~~~~~~~~~~~~~~~~
\KOMAoptions{%
   % titlepage=true %
   titlepage=false %
}%

% ~~~~~~~~~~~~~~~~~~~~~~~~~~~~~~~~~~~~~~~~~~~~~~~~~~~~~~~~~~~~~~~~~~~~~~~~
% head and foot lines
% ~~~~~~~~~~~~~~~~~~~~~~~~~~~~~~~~~~~~~~~~~~~~~~~~~~~~~~~~~~~~~~~~~~~~~~~~

% \pagestyle{scrheadings} % Seite mit Headern
\pagestyle{scrplain} % Seiten ohne Header

% loescht voreingestellte Stile
\clearscrheadings
\clearscrplain
%
% Was steht wo...
% Bei headings:
%   % Oben aussen: Kapitel und Section
%   % Unten aussen: Seitenzahl
%   \ohead{\pagemark}
%   \ihead{\headmark}
%   \ofoot[\pagemark]{} % Außen unten: Seitenzahlen bei plain
% Bei Plain:
\cfoot[\pagemark]{\pagemark} % Mitte unten: Seitenzahlen bei plain


% Angezeigte Abschnitte im Header
% \automark[section]{chapter} %[rechts]{links}
\automark[subsection]{section} %[rechts]{links}

% ~~~~~~~~~~~~~~~~~~~~~~~~~~~~~~~~~~~~~~~~~~~~~~~~~~~~~~~~~~~~~~~~~~~~~~~~
% headings / page opening
% ~~~~~~~~~~~~~~~~~~~~~~~~~~~~~~~~~~~~~~~~~~~~~~~~~~~~~~~~~~~~~~~~~~~~~~~~
\setcounter{secnumdepth}{2}

\KOMAoptions{%
%%%% headings
   % headings=small  % Small Font Size, thin spacing above and below
   % headings=normal % Medium Font Size, medium spacing above and below
   headings=big % Big Font Size, large spacing above and below
}%

% Titelzeile linksbuendig, haengend
\renewcommand*{\raggedsection}{\raggedright} 

% ~~~~~~~~~~~~~~~~~~~~~~~~~~~~~~~~~~~~~~~~~~~~~~~~~~~~~~~~~~~~~~~~~~~~~~~~
% fonts of headings
% ~~~~~~~~~~~~~~~~~~~~~~~~~~~~~~~~~~~~~~~~~~~~~~~~~~~~~~~~~~~~~~~~~~~~~~~~
\setkomafont{sectioning}{\normalfont\sffamily} % \rmfamily
\setkomafont{descriptionlabel}{\itshape}
\setkomafont{pageheadfoot}{\normalfont\normalcolor\small\sffamily}
\setkomafont{pagenumber}{\normalfont\sffamily}

%%% --- Titlepage ---
%\setkomafont{subject}{}
%\setkomafont{subtitle}{}
%\setkomafont{title}{}

% ~~~~~~~~~~~~~~~~~~~~~~~~~~~~~~~~~~~~~~~~~~~~~~~~~~~~~~~~~~~~~~~~~~~~~~~~
% settings and layout of TOC, LOF, 
% ~~~~~~~~~~~~~~~~~~~~~~~~~~~~~~~~~~~~~~~~~~~~~~~~~~~~~~~~~~~~~~~~~~~~~~~~
\setcounter{tocdepth}{3} % Depth of TOC Display

% ~~~~~~~~~~~~~~~~~~~~~~~~~~~~~~~~~~~~~~~~~~~~~~~~~~~~~~~~~~~~~~~~~~~~~~~~
% Tabellen
% ~~~~~~~~~~~~~~~~~~~~~~~~~~~~~~~~~~~~~~~~~~~~~~~~~~~~~~~~~~~~~~~~~~~~~~~~

%%% -| Neue Spaltendefinitionen 'columntypes' |--
%
% Belegte Spaltentypen:
% l - links
% c - zentriert
% r - rechts
% p,m,b  - oben, mittig, unten
% X - tabularx Auto-Spalte

% um Tabellenspalten mit Flattersatz zu setzen, muss \\ vor
% (z.B.) \raggedright geschuetzt werden:
\newcommand{\PreserveBackslash}[1]{\let\temp=\\#1\let\\=\temp}

% Spalten mit Flattersatz und definierte Breite:
% m{} -> mittig
% p{} -> oben
% b{} -> unten
%
% Linksbuendig:
\newcolumntype{v}[1]{>{\PreserveBackslash\RaggedRight\hspace{0pt}}p{#1}}
\newcolumntype{M}[1]{>{\PreserveBackslash\RaggedRight\hspace{0pt}}m{#1}}
% % Rechtsbuendig :
% \newcolumntype{R}[1]{>{\PreserveBackslash\RaggedLeft\hspace{0pt}}m{#1}}
% \newcolumntype{S}[1]{>{\PreserveBackslash\RaggedLeft\hspace{0pt}}p{#1}}
% % Zentriert :
% \newcolumntype{Z}[1]{>{\PreserveBackslash\Centering\hspace{0pt}}m{#1}}
% \newcolumntype{A}[1]{>{\PreserveBackslash\Centering\hspace{0pt}}p{#1}}

\newcolumntype{Y}{>{\PreserveBackslash\RaggedLeft\hspace{0pt}}X}

%-- Einstellungen für Tabellen ----------
\providecommand\tablestyle{%
  \renewcommand{\arraystretch}{1.4} % Groessere Abstaende zwischen Zeilen
  \normalfont\normalsize            %
  \sffamily\small           % Serifenlose und kleine Schrift
  \centering%                       % Tabelle zentrieren
}

%--Einstellungen für Tabellen ----------

\colorlet{tablesubheadcolor}{gray!40}
\colorlet{tableheadcolor}{gray!25}
\colorlet{tableblackheadcolor}{black!60}
\colorlet{tablerowcolor}{gray!15.0}


% ~~~~~~~~~~~~~~~~~~~~~~~~~~~~~~~~~~~~~~~~~~~~~~~~~~~~~~~~~~~~~~~~~~~~~~~~
% pdf packages
% ~~~~~~~~~~~~~~~~~~~~~~~~~~~~~~~~~~~~~~~~~~~~~~~~~~~~~~~~~~~~~~~~~~~~~~~~

% ~~~~~~~~~~~~~~~~~~~~~~~~~~~~~~~~~~~~~~~~~~~~~~~~~~~~~~~~~~~~~~~~~~~~~~~~
% fix remaining problems
% ~~~~~~~~~~~~~~~~~~~~~~~~~~~~~~~~~~~~~~~~~~~~~~~~~~~~~~~~~~~~~~~~~~~~~~~~




% ~~~~~~~~~~~~~~~~~~~~~~~~~~~~~~~~~~~~~~~~~~~~~~~~~~~~~~~~~~~~~~~~~~~~~~~~
% Eigene Befehle
% ~~~~~~~~~~~~~~~~~~~~~~~~~~~~~~~~~~~~~~~~~~~~~~~~~~~~~~~~~~~~~~~~~~~~~~~~
% -- new commands --
\providecommand{\abs}[1]{\lvert#1\rvert}
\providecommand{\Abs}[1]{\left\lvert#1\right\rvert}
\providecommand{\norm}[1]{\left\Vert#1\right\Vert}
\providecommand{\Trace}[1]{\ensuremath{\Tr\{\,#1\,\}}} % Trace /Spur
%

\renewcommand{\d}{\partial\mspace{2mu}} % partial diff
\newcommand{\td}{\,\mathrm{d}}	% total diff

\newcommand{\Ham}{\mathcal{H}}    % Hamilton
\newcommand{\Prob}{\mathscr{P}}    % Hamilton
\newcommand{\unity}{\mathds{1}}   % Real

\renewcommand{\i}{\mathrm{i}}   % imagin�re Einheit



% -- New Operators --
\DeclareMathOperator{\rot}{rot}
\DeclareMathOperator{\grad}{grad}
\DeclareMathOperator{\Tr}{Tr}
\DeclareMathOperator{\const}{const}
\DeclareMathOperator{\e}{e} 			% exponatial Function

\newcommand{\command}[1]{\texttt{\lstinline$#1$}}


% ~~~~~~~~~~~~~~~~~~~~~~~~~~~~~~~~~~~~~~~~~~~~~~~~~~~~~~~~~~~~~~~~~~~~~~~~
% Eigene Befehle
% ~~~~~~~~~~~~~~~~~~~~~~~~~~~~~~~~~~~~~~~~~~~~~~~~~~~~~~~~~~~~~~~~~~~~~~~~
% Silbentrennung hinzufügen als 
% Sil-ben-tren-nung 
\hyphenation{}

\listfiles % schreibt alle verwendeten Dateien in die log Datei

%% Dokument Beginn %%%%%%%%%%%%%%%%%%%%%%%%%%%%%%%%%%%%%%%%%%%%%%%%%%%%%%%%
\begin{document}

% Automatische Titelseite

%\subject{Praktikumsprotokoll}
%\title{ Bash \\ \normalsize Praktikum  1}
%\author{Aljoscha Pörtner \& Max Mustermann}
%\date{09.02.2015}
%\maketitle

% Manuelle Titelseite

\begin{titlepage}
   \mbox{}\vspace{5\baselineskip}\\
   \sffamily\huge
   \centering
   % Titel
   {\Huge Betriebssysteme} \\
  Verzeichnisse \\ \normalsize Praktikum  7
   \vspace{3\baselineskip}\\
   \rmfamily\Large
  Fachhochschule Bielefeld \\
  Campus Minden \\
  Studiengang Informatik
   \vspace{2\baselineskip}\\
\noindent\rule{15cm}{0.4pt}
Beteiligte Personen:
\begin{table}[H]
	\tablestyle
	\rowcolors{1}{tablerowcolor}{white!100}
	\begin{tabular}{*{2}{v{0.45\textwidth}}}
		\hline
		\rowcolor{tableheadcolor}
		\textbf{Name} &
		\textbf{Matrikelnummer} \tabularnewline
		\hline
		%
		
    		Mirko Weidemann Kreitz & 1048290 \tabularnewline
		Oxana Zhurakovskaya  & 130157 \tabularnewline
		Karsten Michael Tymann & 1047529 \tabularnewline
		Yuliia Dobranska & 1093568

	\end{tabular}
\end{table}

   \noindent\rule{15cm}{0.4pt}
      \vspace{1\baselineskip}\\
   \today
\end{titlepage}


\tableofcontents
\newpage
% Testdokumente (auskommentieren!)
%\input{content/hinweis.tex}
%\input{content/demo/demo.tex}
%\input{content/demo/latexexample.tex}

% in diese Datei gehört der Inhalt des Dokumentes:
\section{Aufgabe 1 Typische Verzeichnisstrukturen}
	\begin{itemize}
		\item Geben Sie die typische Verzeichnisstruktur einer aktuellen Linuxdistribution an. Was ist
		in den Verzeichnissen enthalten? Nennen Sie mindestens fünf Beispiele (wie /etc, /usr
		usw.). Finden Sie zusätzlich heraus, wo Log-Dateien gespeichert werden.
		Wo liegen ausgelagerte Inhalte des Hauptspeichers?

		textbf{Antwort}: Ubuntu Verzeichnissystem entspricht Filesystem Hierarchy Standard(FHS):
		\footnote{Quellen:
		\href{http://www.pcwelt.de/ratgeber/So_funktioniert_die_Linux-Ordnerstruktur-Everything_is_a_file-8772939.html}{www.pcwelt.de},
		\href{http://www.selflinux.org/selflinux/html/verzeichnisse_unter_linux01.html}{www.selflinux.org},
		\href{https://jankarres.de/2014/01/debian-linux-verzeichnisbaum-erklaert}{jankarres.de},
		\href{https://wiki.ubuntuusers.de/Verzeichnisstruktur}{wiki.ubuntuusers.de},
		}
		\begin{itemize}
			\item \textbf{/} - Wurzelverzeichnis für alle anderen Linux Verzeichnisse.
			 In der Regel entspricht dieses der Bootpartition (wenn nicht anders beim Installieren eingestellt),
			 deswegen enthält es symbolische Verknüpfungen für initrd.img und vmlinuz,
			 die im Ordner \textit{/boot} liegen
			 \item \textbf{/boot} - beinhaltet alle für den Systemstart(Booten) benötigten Dateien
			 solche wie z.B Kernel \footnote{
			 für Desktop \textit{vmlinuz-versionsnummer-generic},
			 für Server \textit{vmlinuz-versionsnummer-server},
			 für virtuelle Maschinen \textit{vmlinuz-versionsnummer-virtual}
			 }, initiale Ramdisk \footnote{
			 \textit{initrd.img-versionsnummer-generic}/-\textit{server}/-\textit{virtual}}
			 und das Programm für den Memorytest \textit{memtest86.bin}.
			 Außerdem enthält dieser Verzeichnis Ordner \textit{grub/} mit den Bootloader Dateien
			 und dem Ordner \textit{efi/} mit EFI-Programmen
			 Dieses Verzeichnis muss beim Systemstart vorhanden sein.
			\item \textbf{/bin} - Enthält ausführbare Dateien (Programme).
			Es handelt sich bei den Dateien um System-Tools(z.B s cp, echo, mkdir, rm), die von allen Benutzern
			genutzt werden( im Gegensatz zu \textit{/sbin}). - Dieser Ordner darf keine
			weitere Verzeichnisse enthalten. Dieses Verzeichnis muss beim Systemstart vorhanden sein.
			\item \textbf{/dev} - Dieses Verzeichnis beinhaltet alle für den Zugriff
			auf die Geräte erforderliche Dateien (z.B. für  Festplatten, DVD-Laufwerke, Maus, Monitor).
			Hier werden z.B. Festplatenn partitionen eingebunden. Die Dateien werden als Schnittstellen
			für Hardware benutzt.
			(ausgenommen sind jene die mit hot-Plugin eingebunden werden, dafür gibt es den Ordner \textit{/udev})
			Diese Verzeichnis muss beim Systemstart vorhanden sein.
			\item \textbf{/etc} - steht für \textit{``editable text configuration''}.
			Hier liegen systemweit gültige Konfigurations- und Informationsdateien des Basissystems.
			Hier findet man solche Dateien wie z.B. \textit{fstab, hosts, lsb-release, blkid.tab}.
			Diese Konfigurationsdateien können von gleichnamigen Dateien im Homeverzeichnis überschrieben werden.
			er Ordner enthält viele Verzeichnisse mit Konfigurationsdateien.
			Beispiele:
			\begin{itemize}
				\item \textit{/etc/opt}: Verzeichnisse und Konfigurationsdateien für Programme in /opt
				\item \textit{/etc/network}: Verzeichnisse und Konfigurationsdateien des Netzwerkes (interfaces u.s.w.)
				\item \textit{/etc/init.d}: Enthält Start- und Stopskripte
				\item \textit{/etc/hosts} Einstellungen für IP-Adresse auflösung
				\item \textit{/etc/ssh/} enthält SSH Konfigurationsdateien
				\item \textit{/etc/X11} Konfigurationen für für grafische X-Window-Subsystem
			\end{itemize}
			Dieses Verzeichnis muss beim Systemstart vorhanden sein.
			\item \textbf{/home} - Dieser verzeichniss enthält ein Home-ordner jedes Benutzers
			(als name wird Benutzername übernohmen). Hier speichern Benutzer persönliche daten, eigene programme,
			Konfigurationsdateien die nur für den konkreten Benutzer beim Einlogen verfügbar sind.
			Der User hat volle Schreib- und Leserechte für eigene Homeverzeichnis.
			Diese Ordner kann auf eine andere Partition verlegt werden.
			Dieses Verzeichnis muss nicht beim Systemstart vorhanden sein.
			\item \textbf{/lib},\textbf{/lib32},\textbf{/lib64}  - hier liegen wichtige dynamische Bibliotheken des Systems und Kernel-Modules,
			viele werden beim Systemstart benötigt. z.B:
			\begin{itemize}
				\item \textit{/lib/modules} - Kernelmodule
				\item \textit{/lib/udev}: Bibliotheken und Programme für udev
				\item \textit{/lib/linux}-restricted-modules: Speicherort für eingeschränkte Treiber (z.B. Grafikkarte)
			\end{itemize}
			\item \textbf{/lost+found} - Verzeichnis ist in ext2, ext3 und ext4 Dateisystemen vorhanden.
			Ordner beinhaltet Dateien auf der Festplatte,
			die in der Verzeichnisstruktur nicht mehr zugeordnet werden können (bei System-/Programmabstürze oder Hardware-Fehler).
			Bei normal funktinierende System soll derORdner leer bleiben.
			\item \textbf{/media} - Der Verzeichnis wird nur als Einhängerpunkt für Wechseldatenträger
			(solche wie Disketten \textit{/media/floppy}, CD-/DVD-Laufwerke \textit{/media/cdrom},
			\textit{/media/dvd}, Zip-Disks \textit{media/zip}, externe USB-Festplatten).
			Das Verzeichnis mus aud der Gleiche Partition mit dem Root-Verzeichnis liegen.
			\item \textbf{/mnt} - Der Verzeichnis wird als Einhängerpunkt für temporär verfügbare Dateisysteme
			(z.B. eine Windows Partition um auf die dort gespeicherte Daten zugreifen). Theoretisch man könnte
			den den Verzeichnis als Einhängerpunk für Wechseldatenträger nutzen, dies wiederspricht aber den
			FHS Standard.
			\item \textbf{/opt} - Dieser Ordner wird für die manuelle Installation von Programmen genutzt,
			die manuell installiert wurden. Der Verzeichnis ist optional und ist nicht für Den Start des System notwendig.
			Deswegen kann er auch auf andere Partition verlagert werden.
			\item \textbf{/proc} - Der Verzeichnis beinhaltet Prozess- und Systeminformationen und stellt die Schnittstelle zum Kernel dar
			und ist in eigentlichen Sinne spezielles, virtuelles Dateisystem.
			z.B: Beispiele:\textit{version} - Kernelversion, \textit{swaps} Swapspeicherinformationen, \textit{cpuinfo}, \textit{interrupts}, usw.
			Außerden liegen hier Verzeichnisse mit Prozessnumer als Name, die Dateien zu Prozessinformationen enthalten, solche wie
			\textit{status} Prozessstatus, \textit{limits} Limits für Resourcen Verbrauch.
			Dieser Ordner ist nicht FHS Standart aber standartmäsig in Linux distribution Vorhanden. Muss bei Systemstart vorhanden sein.
			\item \textbf{/root} - Das ist der Homeverzeichnis des Superusers (root).
			Hier hat nur Superuser vole Sreib- und Leserechte.
			Dieser Ordnermuss immer Vorhanden sein.
			\item \textbf{/run} - Der Ordner wird für Dateien von laufenden Prozessen genutzt.
			Hier liegen die meisten PID-Files \textit{*.pid} (Process identifier).
			\item \textbf{/sbin } - hier liegen system binaries, die für essentielle Aufgaben der Systemverwaltung
			notwendig sind. Diese Programme können nur von Superuser ausgeführt werden.
			Der Ordner muss auf Rootpartition immer vorhanden sein.
			\item \textbf{/srv }- hier liegen Dateien für  System-Dienste.
			Oft werden hier variable Dateien solche wie Logfiles oder Mails  eines Webservers gespeichert,
			(in Unsrem Ubuntu 14.04 ist dieser Ordner leer).
			Der Verzeichnis ist optional und ist nicht in allen Linuxdistributionen vorhanden.
			\item \textbf{/sys} - der Verzeichnis hat ähnliche funktion wie \textit{/proc} und beinhaltet
 			hauptsächlich Kernelschnittstellen.
		  Der Ordner ist im FHS noch nicht spezifiziert.
			\item \textbf{/tmp} - wird für temporäre Dateien von Programmen genutzt.
			Nach FHS standart wird dieser Ordner nach jedem Neustart bereinigt.
			Jeder Benutzer hat vollen Zugrif nur auf seine eigene temporäre Dateien.
			\item \textbf{/usr} - Diese Verzeichnisstruktur enthält die meisten Systemtools, Bibliotheken und installierten Programme.
			Hier werden Programme gespeichert die von Paketverwaltung installiert wurden.
			Es liegen folgende Unterordner vor:
			\begin{itemize}
				\item \textit{/usr/bin} - hier werden Ausführbare Programme gespeichert
				\item \textit{/usr/include} - hier befinden sich Header-Dateien, die man beim C-Programmierung
				in Source-Code includiert.
				\item \textit{/usr/lib} - weitere Programm-Bibliotheken
				\item \textit{/usr/local} - Der Ordner hat gleiche Struktur wie \textit{/usr} selbst, und kann
				für manuel installierte Programme genutzt werden (wie /opt).
				\item \textit{/usr/sbin} - Hier liegen optionale Systemprogramme
				\item \textit{/usr/share} - hier liegen sich nicht ändernde, architekturunabhängige Dateien.
				\item \textit{/usr/share/applications} - hier findet man Programmstarter, die für Anwendungsmenüs genutzt werden
				\item \textit{/usr/share/man} - beinhaltet Man-pages
			\end{itemize}
			\item \textbf{/var } -wird zu speicherung veränderliche Daten genutzt. Hier liegen nur Vrzeichnisse,
			deren inhalt regelmäsig verändert wird.
			z.B. in \textit{/var/log} Ordner werden ständig Logdateien überschrieben und neu angelegt.


		\end{itemize}
\item Geben Sie die typische Ordnerstruktur von Microsoft Windows an. Nennen Sie analog
zur vorherigen Aufgabe einige beispielhafte Inhalte der jeweiligen Verzeichnisse (wie z.B,
\textit{C:\textbackslash Windows\textbackslash System32}).
\textbf{Antwort:} Windows Ordner: (Pfade/Bezeichnungen können je nach Betriebssystem abweichen)
\footnote{Quellen:}
\begin{itemize}
	\item \textbf{System32}:	\textit{C:\textbackslash Windows\textbackslash System32} - ist das Systemverzeichnis von Windows.

	Inhalte:
	\begin{itemize}
		\item \textit{C:\textbackslash Windows\textbackslash System32\textbackslash Configuration} -
		\item \textit{C:\textbackslash Windows\textbackslash System32\textbackslash config} -
		\item \textit{C:\textbackslash Windows\textbackslash System32\textbackslash de-DE} -
		\item \textit{C:\textbackslash Windows\textbackslash System32\textbackslash GroupPolicy} -
		\item \textit{C:\textbackslash Windows\textbackslash System32\textbackslash Microsoft} -
\end{itemize}
 \item \textbf{Program Files:}	 \textit{C:\textbackslash Program Files(x86)} -
	ist der standard Ordner von Microsoft indem Anwendungen gespeichert werden, die
	nicht zum Betriebssystem gehören. Jede Anwendung erhält zusätzlich ein Unterverzeichnis für
	ihre Ressourcen. Die Bezeichnung kann je nach Betriebssystem den Zusatz (x86) bzw. (x64) mit
	sich führen. Im Ordner \textit{Program Files(x86)} werden 32 Bit Anwendungen und im Ordner
	\textit{Program Files(x64)} 64 Bit Anwendungen standardmäßig gespeichert.

	Inhalte:
	\begin{itemize}
		\item \textit{C:\textbackslash Program Files\textbackslash Adobe}
		\item \textit{C:\textbackslash Program Files\textbackslash MSBuild}
		\item \textit{C:\textbackslash Program Files\textbackslash Microsoft.NET}
		\item \textit{C:\textbackslash Program Files\textbackslash Mozilla Firefox}
		\item \textit{C:\textbackslash Program Files\textbackslash Windows Media Player}
\end{itemize}
\item \textbf{Desktop:}	 \textit{C:\textbackslash Users\textbackslash Username\textbackslash Desktop} -
	existiert zum Einen als Sonderverzeichnis und zum Anderen als virtuelles Verzeichnis.
	Das Sonderverzeichnis enthält die Dateien, die auf dem Desktop des Benutzers gespiechert sind.
	Beim virtuellen Verzeichnis handelt es sich um den Windows-Desktop. Ein Sonderverzeichnis hat einen
	Bezug zu einem echten Verzeichnis auf dem Dateisystem.

	Inhalte:
	\begin{itemize}
	\item \textit{C:\textbackslash Username\textbackslash Desktop\textbackslash4.Semester}
	\item \textit{C:\textbackslash Username\textbackslash Desktop\textbackslash Urlaubsfotos}
	\item \textit{ C:\textbackslash Username\textbackslash Desktop\textbackslash BSPraktikum }
	\item \textit{ C:\textbackslash Username\textbackslash Desktop }
	\item \textit{ C:\textbackslash Username\textbackslash Desktop\textbackslash Softwareprojekt }
\end{itemize}

\item \textbf{Favoriten:}	\textit{C:\textbackslash Users\textbackslash sername\textbackslash Favorites} -
	Hier befinden sich die Favoriten des Benutzers. Links die im Browser als Favoriten
	hinterlegt sind. Favorisierte Dokumente und Verzeichnisse des Nutzers werden im Ordner \textit{Links}
	gepeichert. \textit{C:\textbackslash Users\textbackslash Username\textbackslash Links}

	Inhalte:
	\begin{itemize}
	\item \textit{C:\textbackslash Users\textbackslash Username\textbackslash Favorites\textbackslash Acer\textbackslash Acer.url}
	\item \textit{C:\textbackslash Users\textbackslash Username\textbackslash Favorites\textbackslash Acerv\textbackslash eBay.url}
	\item \textit{‪C:\textbackslash Users\textbackslash Username\textbackslash Favorites\textbackslash Links\textbackslash Acer Zubehoer Shop.url}
	‪\item \textit{C:\textbackslash Users\textbackslash Username\textbackslash Favorites\textbackslash Bing.url}
\end{itemize}

\item 	\textbf{Eigene Dateien:} 	\textit{C:\textbackslash Users\textbackslash Username} -
Hier befinden sich die Dokumente/Verzeichnisse des Benutzers.

Inhalte:
	\begin{itemize}
	\item \textit{C:\textbackslash Users\textbackslash Username\textbackslash Pictures}
	\item \textit{C:\textbackslash Users\textbackslash Username\textbackslash Music}
	\item \textit{C:\textbackslash Users\textbackslash Username\textbackslash Videos}
	\item \textit{C:\textbackslash Users\textbackslash Username\textbackslash Favorites}
	\item \textit{C:\textbackslash Users\textbackslash Username\textbackslash Links}
	\end{itemize}
\end{itemize}
	\end{itemize}

	\subsection{Fazit}

\newpage
\section{Aufgabe 2 Typische Verzeichnisstrukturen}
\subsection{Vorbereitung}
Bearbeiten Sie die folgenden Aufgaben und protokollieren Sie Ihr Vorgehen mithilfe der Vorlage.
Entwickeln Sie ein Programm myls, das den Inhalt von Verzeichnissen ausgibt. Die grundle-
gende Funktion ist in etwa vergleichbar mit dem Shell-Kommando ls.

\subsection{Durchführung}
\begin{itemize}
	\item Zuerst definieren wir die maximale länge für Pfad \command{\#define MAX_PATH 1024}
\item Wir definieren eine Variable die den aktuellen Pfad speichert \command{char path[MAX_PATH];}
	\item Der Name des auszulesenden Verzeichnisses soll dem Programm als Argument übergeben
 werden. Wird kein Verzeichnis angegeben, so wird das lokale Verzeichnis ausgegeben.

 \item Hierzu nutzen wir die Eingabeparameter der main-Methode.\newline
  \command{main(int argc, char *argv[])} Um Parameter annehmen zu können
 nutzen wir die Variable argc um die Anzahl der übergebenen Argumente
 zu zählen sowie das Array argv um die Argumente auslesen zu können.
\item Im nächsten Schritt prüfen wir die Anzahl der übergebenen
Argumente.
\newline
\command{if (argc > 1 && strncmp(argv[1],"-",1)!=0)}
\newline
Wurden Argumente übergeben so ist die Variable argc größer als 1.
Zusätzlich prüfen wir ob das erste übergebene Argument mit einem - anfängt
(mittels strncmp aus header string.h, es wird nur das erste Zeichen geprüft).
Dies würde bedeuten dass der User das listing aus dem lokalen Pfad ausführen
möchte da das - Zeichen das Zeichen für die Parameter ist.
Ist dies der Fall so wird der else-Teil ausgeführt. Dieser wird auch ausgeführt
wenn keine Argumente übergeben werden.
\command{getcwd(path, MAX_PATH);} Im else-Teil wird dann das aktuelle
Working Directory mittels getcwd ermittelt und dem Array path übergeben.
Wurde ein Pfad übergeben so betrachten wir den If Block.
\command{strcpy(path, argv[1]);} Hier überweisen wir dem path Array den vom
User übergebenen Pfad.
\item Danach bauen wir eine Abfrage ein ob ein übergebener Pfad mit einem
/ endet. Ist dies nicht der Fall so hängen wir eines an.
Hierzu ermitteln wir zunächst die Länge des Pfades:
\command{int len = strlen(path);}
Dann erzeugen wir einen Pointer der auf den letzten Index des Arrays
zeigt \command{const char *last = &path[len - 1];}
Nun prüfen wir ob dieses Zeichen ein / ist
\command{if (strcmp(last, ``/'') != 0)}
Ist dies nicht der Fall so hängen wir eins an.
\command{strcat(path,"/");}
Diese Methoden der String-Manipulation entstammen dem string.h Header.
\item Wir definieren eine Variable \command{int c;} wo wir eine einzige Option
aus der übergebenen Options Liste speichern
\item Um die Optionen für myls zu speichern und weiter auszuwerten legen wir drei Variablen an
\begin{itemize}
	\item \command{int aoption = 0} für Option -a (gültige Werte 0 oder 1)
	\item \command{int loption = 0;} für -l und g Option (gültige Werte 0, 1, 2)
	\item \command{int ooption = 0;} für -o Option (gültige Werte 0 oder 1)
\end{itemize}

\item Wir nutzen die Funktion \command{getopt(argc, argv, ``algo'')} aus dem Header \command{unistd.h},
die das Auslesen der Optionen aus dem Argument-string erleichtert. Als ersten und zweiten Argument
werden die Argumente die die main Methode erhält an \command{getopt()} übergeben,
das dritte Argument übergeben wir als String aus allen gültigen Argumenten,
nach denen durchgesucht wird.
Wenn es keine Argumente in \command{argv0} gibt, dann liefert die Methode den Wert -1 zurück.
\item Den Rückgabewert von \command{getopt} nutzen wir als Abbruch Bedingung für die while-Schleife

\command{while ((c = getopt(argc, argv, ``algo'')) != -1)}
\item In der while-Schleife prüfen wir mit switch-case welche Option ausgelesen wurde,
und ändern die Werte von den oben genannten Variablen: a-,l-,ooption.
Wird beispielsweise ein a gelesen so wird der zutreffende case ausgeführt
und das aoption Flag auf 1 gesetzt.
Bei einem gelesenen l wird die loption auf 1 gesetzt.
Wird ein g gesetzt so wird die loption auf 2 gesetzt, folglich erkennen wir
dass das längere Ausgabeformat aber ohne UserID gewünscht ist.
Wird ein o gesetzt so setzen wir die ooption auf 1. Zudem prüfen wir
ob das loption Flag bereits gesetzt wurde. Wurde beispielsweise nur ein
o als Parameter übergeben so müssen wir das loption Flag auf 1 setzen damit
wir wissen dass das längere Ausgabeformat gewünscht ist.
Die ooption sorgt dafür dass die GroupID ausgeblendet werden soll.
Kombinationen der Parameter sind somit möglich.
\item Danach rufen wir die Funktion \command{readPath(path, aoption, loption, ooption);}
auf. Wir übergeben hier den Pfad sowie die 3 Flags.

 \begin{lstlisting}
	 int main(int argc, char *argv[]) {
	if (argc > 1 && strncmp(argv[1],"-",1)!=0) {
		strcpy(path, argv[1]);
	} else {
		getcwd(path, MAX_PATH);
	}
	int len = strlen(path);
	const char *last = &path[len - 1];
	if (strcmp(last, "/") != 0) {
		strcat(path,"/");
	}
	printf("input path: %s", path);
	int c;
	int aoption = 0;
	int loption = 0;
	int ooption = 0;
	while ((c = getopt(argc, argv, "algo")) != -1) {
		switch (c) {
		case 'a':
			aoption = 1;
			break;
		case 'l':
			loption = 1;
			break;
		case 'g':
			loption = 2;
			break;
		case 'o':
			if(loption==0)
				loption=1;
			ooption = 1;
			break;
		}
	}

	readPath(path, aoption, loption, ooption);
	return 0;
}
 \end{lstlisting}
\item Um den Verzeichnis Inhalt und Informationen auszulesen definieren wir eine Funktion

\command{void *readPath(char *path, int aoption, int loption, int ooption)}

\item Die Funktion nimmt als erstes Argument den Verzeichnisnamen, deren Inhalt ausgelesen wird,
die 3 folgenden Argumente sind die FLags.
\item Um Ordner Information zu lesen nutzen wir die Bibliothek \command{dirent.h},
\item Wir legen uns eine Variable für den aufgelösten Pfad an, mit der
bereits definierten Größe des maximalen Pfades
\command{char resolved_path[MAX_PATH];}
Außerdem definieren wir eine Variable  \command{DIR *dir = NULL;},
die den Directory Stream beinhalten wird.

Um den Ordner Inhalt aus dem directory stream zu lesen,definieren wir eine Variable

\command{struct dirent *dptr = NULL;}

Und eine Char Variable die lediglich einen Punkt hält, die wir zum Vergleich
nutzen werden.
\command{char *dot = ``.'';}

\item Um den absoluten Pfad zu erhalten nutzen wir die Funktion
\newline
\command{realpath((char*)path, resolved_path)}
\newline
aus der Standartbibliothek,
als Parameter übergeben wir den vom Benutzer übergebenen Pfad \command{path}
und unsere Variable \command{resolved_path}, die das Ergebnis erhalten wird,
dabei casten wir die Variable \command{path} zu einem char pointer.
Die Funktion wird einen \command{NULL} pointer zurückliefern, falls beim Pfadname auflösen ein Fehler auftritt,
deswegen können wir das Ergebnis der Variable \command{realpath} in der If-Abfrage überprüfen,
und falls etwas mit dem Pfad nicht stimmt, kann die Funktion \command{readPath} ihre Arbeit abbrechen.
\item Wenn der Pfad erfolgreich aufgelöst wurde können wir weiter vorgehen.
\item Wir öffnen den directory stream mit der Funktion \command{opendir(resolved_path)},
die Funktion wird einen \command{NULL} Pointer zurückliefern wenn ein Fehler beim Öffnen auftritt.
Wir fragen dsd Resultat ebenfalls in der If-Abfrage ab, wenn der Ordner erfolgreich geöffnet wurde,
kann man weiter vorgehen, ansonsten muss die Funktion \command{readPath} ihr Arbeit abbrechen.

\command{if ((dir = opendir(resolved_path)))}

\item Nun kann man mit der While-Schleife durch die einzelnen Einträge im
directory stream iterieren,
dabei hilft uns die Variable \command{dptr}, die bei jeder Iteration auf
den nächsten Eintrag zeigt.
Um den nächsten Eintrag auszulesen, benutzen wir die Funktion \command{readdir(dir)},
die als Parameter einen directory stream annimmt

\command{while ((dptr = readdir(dir)))}

\item Innerhalb der while-Schleife prüfen wir ob das aoption Flag NICHT gesetzt ist
\command{if (!aoption)}

Wir schließen die Dateien die mit dem Namen \textit{.} beginnen aus,
indem wir den Namen des aktuellen Eintrags mit der Variable \command{dot} vergleichen.
Und wenn es sich um diese Dateien handelt, geht die while schleife
ohne weiteres Vorgehen zur nächsten Iteration.

\command{if (strncmp(dptr->d_name, dot, 1) == 0) continue;}

\item Ist das a-Flag also gesetzt so nehmen wir auch die Versteckten Dateien
mit.

\item Nun prüfen wir ob das l-Flag gesetzt wurde \command{if (loption!=0)}

Ist dies der Fall, so legen wir uns eine Struktur vom Typ stat an, wo
wir Dateiinformationen speichern können.
\command{struct stat lstruct;}

Die Methode \command{printlstruct(dptr->d_name,loption, ooption);} wird
später erklärt. Sie erhält als Eingabeparameter den aktuellen File-Namen, sowie
die l- und o-Flags.

\item Mittels \command{lstat(dptr->d_name, &lstruct)} lesen wir die Information von der aktuellen Datei aus
und Speichern diese in der vorher definierten \command{lstruct} Variable.
Dabei folgt die Funktion im Gegensatz zur Funktion \command{stat()} nicht dem symbolic link,
sondern es wird die Information über den Link selbst ausgelesen und nicht über die referenzierte Datei.
Wenn die Information erfolgreich ausgelesen wurde, liefert die funktion den Wert 0 zurück,
sonst den Wert -1
\item Wir nutzen den Rückgabewert von der Funktion  \command{lstat} in der if abfrage,
um zu prüfen ob der lstat Aufruf funktioniert hat.
\item Nun wird geprüft, ob es sich bei der Datei um eine ausführbare Datei handelt.
Ist dies der Fall so soll die Datei rot ausgegeben werden.
\item Dazu prüfen wir innerhalb der if-Abfrage ob der User, die Gruppe oder
oder Andere execute-Rechte haben. Zusätzlich prüfen wir ob es eine
Ausführbare Datei ist \command{S_IEXEC}. Dieses Attribut ist allerdings
bereits durch \command{S_IXUSR} abgelöst worden.

Um also auf diese Rechte vergleichen zu können lesen wir aus dem
struct den mode\_t aus in denen diese Flags gesetzt sind.
Der Vergleich ob das Flag gesetzt ist erfolgt über den logischen
UND-Operator.
\command{(lstruct.st_mode & S_IXUSR)}

Diese Abfrage zieht sich für die anderen Abfragen so durch.
\begin{lstlisting}
	if((lstat(dptr->d_name, &lstruct) == 0 &&
	((lstruct.st_mode & S_IXUSR) ||
	(lstruct.st_mode & S_IXGRP) ||
	(lstruct.st_mode & S_IXOTH) ||
	(lstruct.st_mode & S_IEXEC) ))
\end{lstlisting}

\item Ist also der Aufruf von lstat geglückt, und ist die Datei eine ausführbare
Datei, so legen wir den Farb Code für die nächste Ausgabe fest.

\command{printf("\\033[0;31;1m");}

Die Eröffnung der Sequenz ist dabei die \command{\\033[}.

Es folgt die Hintergrundfarbe die wir gerne bei schwarz belassen \command{0;}.

Nun die Schriftfarbe \command{31;}. Der Farbcode entspricht Rot.

Es folgt die Vordergrundfarbe \command{1m} die keinen Effekt hat.
Das m schließt die Sequenz ab.
Nun ist jeder folgende Output Rot.

\item Um die Dateien mit Endung \textit{.c} herauszufiltern, und diese anschließend grün zu färben
ermitteln wir zuerst die Länge des aktuellen Dateinamens und speichern diese in einer Variable

\command{int len = strlen(dptr->d_name);}

und wir erzeugen einen Pointer zum vorletzten Zeichen im Dateinamen

\command{const char *last_two = &dptr->d_name[len - 2];}.

Anschließend prüfen wir mit der Funktion \command{strcmp(last_two, ``.c'')}
anhand der letzten zwei Zeichen ob es sich um die Endung ``.c'' handelt.
Wenn dies der Fall wird die Ausgabe grün gefärbt \command{printf("\\033[0;32;1m");}

\item Außerhalb des if-Blocks wird dann der Name der aktuellen Datei
ausgegeben \command{printf(``\%s\n'', dptr->d_name);}.

\item Falls der loption Flag gesetzt war, so muss nun die Textfarbe wieder
zurückgesetzt werden \command{if (loption!=0) printf("\033[0;0;0m");}.

\item Ist die while-Schleife durchgelaufen so schließen wir den directory
stream \command{closedir(dir);}.


\begin{lstlisting}[language=C]
void *readPath(char *path, int aoption, int loption, int ooption) {
char resolved_path[MAX\_PATH];
DIR *dir = NULL;
struct dirent *dptr = NULL;
char *dot = ``.'';
if (realpath(path, resolved_path)) {
printf("resolved_path: %s\n", resolved_path);
if ((dir = opendir(resolved_path))) {
while ((dptr = readdir(dir))) {
if (!aoption) {
	if (strncmp(dptr->d_name, dot, 1) == 0) {
		continue;
	}
}
if (loption!=0) {
	struct stat lstruct;

	printlstruct(dptr->d_name,loption, ooption);

	if(lstat(dptr->d_name, &lstruct) == 0 &&

	((lstruct.st_mode & S_IXUSR) ||
	(lstruct.st_mode & S_IXGRP) ||

	(lstruct.st_mode & S_IXOTH) ||
	(lstruct.st_mode & S_IEXEC) )){
		printf("\\033[0;31;1m");
	}

	int len = strlen(dptr->d_name);
	const char *last_two = &dptr->d_name[len - 2];
	if (strcmp(last_two, ``.c'') == 0) {
		printf("\\033[0;32;1m");
	}

}
printf("\%s\n", dptr->d_name);
if (loption!=0)
	printf("\\033[0;0;0m");
	}
closedir(dir);
}
}
return NULL;
}
\end{lstlisting}
\item Um die ausführlichen Informationen über den File auszugeben, definieren wir eine Funktion
\newline
\command{printlstruct(char * filename, int loption, int ooption)}
\newline
\item Um auf die Fileattribute zugreifen zu können definieren wir die Struktur
\newline \command{struct stat lstruct;}
\item Wir brauchen auch einen vollen Pfad für die Dateinamen für die Funktion \command{lstat}
Dafür definieren wir eine Variable \command{char fullpath[MAX_PATH];} und dann
bauen wir den aus der in der Variable \command{path} vorhandenen Pfad zum Verzeichnis
und aus dem aktuellen Dateinamen zusammen.
\begin{lstlisting}
	strcpy(fullpath,path);
	strcat(fullpath,filename);
\end{lstlisting}
\item Diese Pfad übergeben wir zusammen mit \command{lstruct} an die Funktion
\command{lstat(fullpath,&lstruct);}
\item Anschließend überprüfen wir ob die Lese-Schreib-Execute Rechte für den
Owner, die Gruppe und Andere gesetzt sind. Hierzu prüfen wir die einzelnen
Flags und geben dann bei Erfolg oder Misserfolg das entsprechend Zeichen aus.

\begin{lstlisting}
printf( (lstruct.st_mode & S_IRUSR) ? "r" : "-");
printf( (lstruct.st_mode & S_IWUSR) ? "w" : "-");
printf( (lstruct.st_mode & S_IXUSR) ? "x" : "-");
printf( (lstruct.st_mode & S_IRGRP) ? "r" : "-");
printf( (lstruct.st_mode & S_IWGRP) ? "w" : "-");
printf( (lstruct.st_mode & S_IXGRP) ? "x" : "-");
printf( (lstruct.st_mode & S_IROTH) ? "r" : "-");
printf( (lstruct.st_mode & S_IWOTH) ? "w" : "-");
printf( (lstruct.st_mode & S_IXOTH) ? "x\t" : "-\t");
\end{lstlisting}

\item Es folgt die Ausgabe der verschiedenen Strukturelemente der stat
Struktur.
\item Nun geben wir die Anzahl der Links auf die Datei aus.
\newline
\command{printf("\%ld",(long) lstruct.st_nlink);}
\newline
\item Für die UserID des Dateibesitzers prüfen wir ob das g Flag nicht
gesetzt wurde. Dann soll es ausgeführt werden.
\newline
\command{if(loption!=2) printf("\\t\%ld",(long) lstruct.st_uid);}
\newline
\item Nun prüfen wir ob das o Flag gesetzt wurde. Ist dies nicht
der Fall so geben wir die GroupID des Dateibesitzers aus.
\newline
\command{if(ooption==0)	printf("\\t\%ld",(long) lstruct.st_gid);}
\newline

\item Es folgt die Ausgabe der Dateigröße in Bytes
\newline
\command{printf("\\t\%lld",(long long) lstruct.st_size);}
\newline

\item Als nächstes folgen die Ausgaben für die Zeitpunkte des letzten Zugriffs,
der letzten Modifikation und der letzten Statusänderung.

\begin{lstlisting}
	printf("\t%s",ctime(&lstruct.st_atime));
	printf("\t%s", ctime(&lstruct.st_mtime));
	printf("\t%s", ctime(&lstruct.st_ctime));
\end{lstlisting}

\item Abschließend die Ausgabe der I/O Block Größe.
\newline
\command{printf("\\t\%ld ",(long) lstruct.st_blksize);}
\newline


\begin{lstlisting}
void printlstruct(char * filename, int loption, int ooption){
	struct stat lstruct;
	char fullpath[MAX_PATH];
	strcpy(fullpath,path);

	strcat(fullpath,filename);
	lstat(fullpath,&lstruct);

	printf( (lstruct.st_mode & S_IRUSR) ? "r" : "-");
	printf( (lstruct.st_mode & S_IWUSR) ? "w" : "-");
	printf( (lstruct.st_mode & S_IXUSR) ? "x" : "-");
	printf( (lstruct.st_mode & S_IRGRP) ? "r" : "-");
	printf( (lstruct.st_mode & S_IWGRP) ? "w" : "-");
	printf( (lstruct.st_mode & S_IXGRP) ? "x" : "-");
	printf( (lstruct.st_mode & S_IROTH) ? "r" : "-");
	printf( (lstruct.st_mode & S_IWOTH) ? "w" : "-");
	printf( (lstruct.st_mode & S_IXOTH) ? "x\t" : "-\t");

	printf("%ld",(long) lstruct.st_nlink);
	if(loption!=2)
		printf("\t%ld",(long) lstruct.st_uid);
	if(ooption==0)
		printf("\t%ld",(long) lstruct.st_gid);
	printf("\t%lld",(long long) lstruct.st_size);
	printf("\t%s",ctime(&lstruct.st_atime));
	printf("\t%s", ctime(&lstruct.st_mtime));
	printf("\t%s", ctime(&lstruct.st_ctime));
	printf("\t%ld ",(long) lstruct.st_blksize);
}
\end{lstlisting}

\end{itemize}
Gesamte Code:
\begin{lstlisting}
/*
* myls.c
*
*/
#include <stdio.h>
#include <stdlib.h>
#include <string.h>
#include <dirent.h>
#include <unistd.h>
#include <time.h>
#include <sys/types.h>
#include <sys/stat.h>
#include <unistd.h>

#define MAX_PATH 1024

char path[MAX_PATH];

void printlstruct(char * filename, int, int);
void *readPath(char *path, int, int, int);

int main(int argc, char *argv[]) {
if (argc > 1 && strncmp(argv[1],"-",1)!=0) {
strcpy(path, argv[1]);
} else {
getcwd(path, MAX_PATH);

}
int len = strlen(path);
const char *last = &path[len - 1];
if (strcmp(last, "/") != 0) {
strcat(path,"/");
}


printf("input path: %s", path);
int c;
int aoption = 0;
int loption = 0;
int ooption = 0;
while ((c = getopt(argc, argv, "algo")) != -1) {
switch (c) {
case 'a':
aoption = 1;
break;
case 'l':
loption = 1;
break;
case 'g':
loption = 2;
break;
case 'o':
if(loption==0)
loption=1;
ooption = 1;
break;
}
}

readPath(path, aoption, loption, ooption);
return 0;
}
void *readPath(char *path, int aoption, int loption, int ooption) {
char resolved_path[MAX_PATH];
DIR *dir = NULL;
struct dirent *dptr = NULL;
char *dot = ".";
if (realpath(path, resolved_path)) {
printf("resolved_path: %s\n", resolved_path);
if ((dir = opendir(resolved_path))) {
while ((dptr = readdir(dir))) {
if (!aoption) {
if (strncmp(dptr->d_name, dot, 1) == 0) {
continue;
}
}
if (loption!=0) {
struct stat lstruct;

printlstruct(dptr->d_name,loption, ooption);

if(lstat(dptr->d_name, &lstruct) == 0 &&
((lstruct.st_mode & S_IXUSR) ||
	(lstruct.st_mode & S_IXGRP) ||
	(lstruct.st_mode & S_IXOTH) ||
	(lstruct.st_mode & S_IEXEC) )){
printf("\033[0;31;1m");
}

int len = strlen(dptr->d_name);
const char *last_two = &dptr->d_name[len - 2];
if (strcmp(last_two, ".c") == 0) {
printf("\033[0;32;1m");
}

}
printf("%s\n", dptr->d_name);
if (loption!=0)
printf("\033[0;0;0m");

}
closedir(dir);
}
}
return NULL;

}

void printlstruct(char * filename, int loption, int ooption){
struct stat lstruct;
char fullpath[MAX_PATH];
strcpy(fullpath,path);

strcat(fullpath,filename);
lstat(fullpath,&lstruct);

printf( (lstruct.st_mode & S_IRUSR) ? "r" : "-");
printf( (lstruct.st_mode & S_IWUSR) ? "w" : "-");
printf( (lstruct.st_mode & S_IXUSR) ? "x" : "-");
printf( (lstruct.st_mode & S_IRGRP) ? "r" : "-");
printf( (lstruct.st_mode & S_IWGRP) ? "w" : "-");
printf( (lstruct.st_mode & S_IXGRP) ? "x" : "-");
printf( (lstruct.st_mode & S_IROTH) ? "r" : "-");
printf( (lstruct.st_mode & S_IWOTH) ? "w" : "-");
printf( (lstruct.st_mode & S_IXOTH) ? "x\t" : "-\t");

printf("%ld",(long) lstruct.st_nlink);
if(loption!=2)
printf("\t%ld",(long) lstruct.st_uid);
if(ooption==0)
printf("\t%ld",(long) lstruct.st_gid);
printf("\t%lld",(long long) lstruct.st_size);
printf("\t%s",ctime(&lstruct.st_atime));
printf("\t%s", ctime(&lstruct.st_mtime));
printf("\t%s", ctime(&lstruct.st_ctime));
printf("\t%ld ",(long) lstruct.st_blksize);
}
\end{lstlisting}




\subsection{Fazit}


\end{document}
%% Dokument ENDE %%%%%%%%%%%%%%%%%%%%%%%%%%%%%%%%%%%%%%%%%%%%%%%%%%%%%%%%%%
