
	\begin{itemize}
		\item Geben Sie die typische Verzeichnisstruktur einer aktuellen Linuxdistribution an. Was ist
		in den Verzeichnissen enthalten? Nennen Sie mindestens fünf Beispiele (wie /etc, /usr
		usw.). Finden Sie zus ̈atzlich heraus, wo Log-Dateien gespeichert werden.
		Wo liegen ausgelagerte Inhalte des Hauptspeichers?

		Antwort: .

		Hierarhie:
		\begin{itemize}
			\item / - Rootverzeichnis für alle andere Linux Verzeichnisse,
			\item /bin - binaries - Ausführbare dateien (Programme),
			\item /boot -
			\item /dev -
			\item /etc -
			\item /home -
			\item /lib -
			\item /lost+found -
			\item /media -
			\item /mnt -
			\item /opt -
			\item /proc -
			\item /root -
			\item /run -
			\item /sbin -
			\item /srv -
			\item /sys -
			\item /tmp -
			\item /usr -
			\item /var -

		\end{itemize}



	\end{itemize}

	\subsection{Fazit}

\newpage
\section{Aufgabe 1 Typische Verzeichnisstrukturen}
\subsection{Vorbereitung}
Bearbeiten Sie die folgenden Aufgaben und protokollieren Sie Ihr Vorgehen mithilfe der Vorlage.
Entwickeln Sie ein Programm myls, das den Inhalt von Verzeichnissen ausgibt. Die grundle-
gende Funktion ist in etwa vergleichbar mit dem Shell-Kommando ls.

\subsection{Durchführung}

\subsection{Fazit}
