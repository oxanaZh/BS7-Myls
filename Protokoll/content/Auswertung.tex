\section{Aufgabe 1 Typische Verzeichnisstrukturen}
	\begin{itemize}
		\item Geben Sie die typische Verzeichnisstruktur einer aktuellen Linuxdistribution an. Was ist
		in den Verzeichnissen enthalten? Nennen Sie mindestens fünf Beispiele (wie /etc, /usr
		usw.). Finden Sie zus ̈atzlich heraus, wo Log-Dateien gespeichert werden.
		Wo liegen ausgelagerte Inhalte des Hauptspeichers?

		Antwort: Ubuntu 16.04 Verzeichnissystem entspricht Filesystem Hierarchy Standard(FHS):

		\begin{itemize}
			\item \textbf{/} - Rootverzeichnis für alle anderen Linux Verzeichnisse.
			 In der Regel entspricht der Bootpartition (wenn nicht anders beim Installieren eingestellt wurde),
			 deswegen enthält  symbolischen Verknüpfungen für initrd.img und vmlinuz,
			 die in Ordner \textit{/boot} liegen
			 \item \textbf{/boot} - beinhaltet alle für Systemstart(Booten) benötigte Dateien
			 solche wie z.B Kernel \footnote{
			 für Desktop \textit{vmlinuz-versionsnummer-generic},
			 für Server \textit{vmlinuz-versionsnummer-server},
			 für virtuelle Maschinen \textit{vmlinuz-versionsnummer-virtual}
			 }, initiale Ramdisk \footnote{
			 \textit{initrd.img-versionsnummer-generic}/-\textit{server}/-\textit{virtual}}
			 und das Programm für den Memorytest \textit{memtest86.bin}.
			 Außerdem enthält dieser Verzeichnis Ordner \textit{grub/} mit Bootloader dateien
			 und Ordner \textit{efi/} mit EFI-Programmen
			 Diese Verzeichnis muss beim Systemstart vorhanden sein.
			\item \textbf{/bin} - Enthält ausführbare Dateien (Programme).
			Es handelt sich bei den Dateien um System-Tools(soche wie), die von allen Benutzern
			genutzt werden( im gegensatzt zu \textit{/sbin}). -Diese Ordner darf keine
			weitere Verzeichnisse enthalten. Diese Verzeichnis muss beim Systemstart vorhanden sein.
			\item \textbf{/dev} - Diese Verzeichnis beinhaltet alle für den zugriff
			auf die Geräte erforderliche Dateien (z.B. für  Festplatten und DVD-Laufwerke).
			(ausgenohmen die mit hot-Plugin eingebunden werden, dafür gibt es Ordner \textit{/udev})
			Diese Verzeichnis muss beim Systemstart vorhanden sein.
			\item \textbf{/etc} - steht für \textit{``editable text configuration''}.
			Hier liegen Konfigurations- und Informationsdateien des Basissystems.
			Hier findet man solche dateien wie z.B. \textit{fstab, hosts, lsb-release, blkid.tab}.
			Diese Konfigurationsdateien können von gleichnamigen Dateien in Homeverzeichnis überschrieben werden.
			Diese Verzeichnis muss beim Systemstart vorhanden sein.
			\item \textbf{/home} -
			\item \textbf{/lib} -
			\item \textbf{/lost+found} -
			\item \textbf{/media} -
			\item \textbf{/mnt} -
			\item \textbf{/opt} -
			\item \textbf{/proc} -
			\item \textbf{/root} -
			\item \textbf{/run} -
			\item \textbf{/sbin }-
			\item \textbf{/srv }-
			\item \textbf{/sys} -
			\item \textbf{/tmp} -
			\item \textbf{/usr} -
			\item \textbf{/var }-

		\end{itemize}



	\end{itemize}

	\subsection{Fazit}

\newpage
\section{Aufgabe 2 Typische Verzeichnisstrukturen}
\subsection{Vorbereitung}
Bearbeiten Sie die folgenden Aufgaben und protokollieren Sie Ihr Vorgehen mithilfe der Vorlage.
Entwickeln Sie ein Programm myls, das den Inhalt von Verzeichnissen ausgibt. Die grundle-
gende Funktion ist in etwa vergleichbar mit dem Shell-Kommando ls.

\subsection{Durchführung}

\subsection{Fazit}
